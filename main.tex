%-------------------------
% Resume in Latex
% Author : Aras Gungore
% License : MIT
%------------------------

\documentclass[letterpaper,11pt]{article}

\usepackage{latexsym}
\usepackage[empty]{fullpage}
\usepackage{titlesec}
\usepackage{marvosym}
\usepackage[usenames,dvipsnames]{color}
\usepackage{verbatim}
\usepackage{enumitem}
\usepackage[hidelinks]{hyperref}
\usepackage{fancyhdr}
\usepackage[english]{babel}
\usepackage{tabularx}
\usepackage{hyphenat}
\usepackage{fontawesome}
\usepackage{qrcode}
\usepackage{multicol}
\input{glyphtounicode}


%---------- FONT OPTIONS ----------
% sans-serif
% \usepackage[sfdefault]{FiraSans}
% \usepackage[sfdefault]{roboto}
% \usepackage[sfdefault]{noto-sans}
% \usepackage[default]{sourcesanspro}

% serif
% \usepackage{CormorantGaramond}
% \usepackage{charter}

\pagestyle{fancy}
\fancyhf{} % clear all header and footer fields
\fancyfoot{}
\renewcommand{\headrulewidth}{0pt}
\renewcommand{\footrulewidth}{0pt}

% Adjust margins
\addtolength{\oddsidemargin}{-0.5in}
\addtolength{\evensidemargin}{-0.5in}
\addtolength{\textwidth}{1in}
\addtolength{\topmargin}{-.5in}
\addtolength{\textheight}{1.0in}

\urlstyle{same}

\raggedbottom
\raggedright
\setlength{\tabcolsep}{0in}

% Sections formatting
\titleformat{\section}{
  \vspace{-4pt}\scshape\raggedright\large
}{}{0em}{}[\color{black}\titlerule \vspace{-5pt}]

% Ensure that generate pdf is machine readable/ATS parsable
\pdfgentounicode=1

%-------------------------
% Custom commands

\newcommand{\resumeItem}[1]{
  \item\small{
    {#1 \vspace{-2pt}}
  }
}


\newcommand{\resumeSubheading}[4]{
  \vspace{-2pt}\item
    \begin{tabular*}{0.97\textwidth}[t]{l@{\extracolsep{\fill}}r}
      \textbf{#1} & #2 \\
      \textit{\small#3} & \textit{\small #4} \\
    \end{tabular*}\vspace{-7pt}
}


\newcommand{\resumeSubSubheading}[2]{
    \vspace{-2pt}\item
    \begin{tabular*}{0.97\textwidth}{l@{\extracolsep{\fill}}r}
      \textit{\small#1} & \textit{\small #2} \\
    \end{tabular*}\vspace{-7pt}
}


\newcommand{\resumeEducationHeading}[6]{
  \vspace{-2pt}\item
    \begin{tabular*}{0.97\textwidth}[t]{l@{\extracolsep{\fill}}r}
      \textbf{#1} & #2 \\
      \textit{\small#3} & \textit{\small #4} \\
      \textit{\small#5} & \textit{\small #6} \\
    \end{tabular*}\vspace{-5pt}
}


\newcommand{\resumeProjectHeading}[2]{
    \vspace{-2pt}\item
    \begin{tabular*}{0.97\textwidth}{l@{\extracolsep{\fill}}r}
      \small#1 & #2 \\
    \end{tabular*}\vspace{-7pt}
}


\newcommand{\resumeOrganizationHeading}[4]{
  \vspace{-2pt}\item
    \begin{tabular*}{0.97\textwidth}[t]{l@{\extracolsep{\fill}}r}
      \textbf{#1} & \textit{\small #2} \\
      \textit{\small#3}
    \end{tabular*}\vspace{-7pt}
}

\newcommand{\resumeSubItem}[1]{\resumeItem{#1}\vspace{-4pt}}

\renewcommand\labelitemii{$\vcenter{\hbox{\tiny$\bullet$}}$}

\newcommand{\resumeSubHeadingListStart}{\begin{itemize}[leftmargin=0.15in, label={}]}
\newcommand{\resumeSubHeadingListEnd}{\end{itemize}}
\newcommand{\resumeItemListStart}{\begin{itemize}}
\newcommand{\resumeItemListEnd}{\end{itemize}\vspace{-5pt}}

%-------------------------------------------
%%%%%%  RESUME STARTS HERE  %%%%%%


\begin{document}

%---------- HEADING ----------

\begin{center}
    \textbf{\Huge \scshape Eduardo Hernández Valdez } \\ \vspace{3pt}
    \small
    \faMobile \hspace{.5pt} \href{tel:528112373193}{+52 81 1237 3193}
    $|$
    \faAt \hspace{.5pt} \href{mailto:eduarch42@protonmail.com}{eduarch42@protonmail.com}
    $|$
    \faLinkedinSquare \hspace{.5pt} \href{https://www.linkedin.com/in/eduardohdz42}{LinkedIn}
    $|$
    \faGithub \hspace{.5pt} \href{https://github.com/saucesaft}{GitHub}
    $|$
    \faMapMarker \hspace{.5pt} \href{https://goo.gl/maps/xJkfKzoSjL6UDZhN9}{Mty, MX}

\end{center}



%----------- EDUCATION -----------

\section{Education}
  \vspace{3pt}
  \resumeSubHeadingListStart
    
    \resumeSubheading
      {Instituto Tecnológico y de Estudios Superiores de Monterrey} {Monterrey, México}
      {Engineering in Robotics and Digital Systems; \textbf{Avg 90.26/100}}{June 2026}

    \resumeSubheading
      {PrepaTec Eugenio Garza Lagüera }{Monterrey, México}
      {High School Diploma;   \textbf{90/100}}{Aug 2019 \textbf{--} Jun 2022}
    
  \resumeSubHeadingListEnd

%----------- RESEARCH EXPERIENCE -----------

\section{Research Experience}
  \vspace{3pt}
  \resumeSubHeadingListStart
  
    \resumeSubheading
      { VantTec - Self Driving Vehicle } { Monterrey, México }
      { Embedded and Electronics Research and Development Member}{ Jan 2023 \textbf{--} Present }
        \resumeItemListStart
            \resumeItem{Worked on pcb design regarding the vehicle's security and redundancy system requirements, using KiCAD. Translated prototypes to pcb boards needed as functional systems inside the vehicle. (STM32, Electric circuitry) }
            \resumeItem{Created a "from-scratch" python library to receive data from CANbus based sensors, such as rotary encoders and radars. To be used for the vehicle as an interface through ROS 2. (Briter Encoder, Ainstein T-79 Radar)}
	    \resumeItem{Create documentation regarding the vehicles already existing subsystems in order to integrate them into the autonomous arrangement, and assure the project's future continuation.}
        \resumeItemListEnd
    
    \resumeSubheading
      { VantTec - Unmanned Surface Vehicle} {Monterrey, México}
      { Perception Research and Development Member}{ Aug 2022 \textbf{--} Present }
        \resumeItemListStart
            \resumeItem{Created datasets for the training of a neural network to detect and classify objects in water, aiding to the vehicle's self navigation for the Roboboat competition, and published the results as an academic paper, participating at a recognized engineering expo.}
	    \resumeItem{Developed on-demand computer vision algorithms regarding the vehicle's needs in each section of the competition using OpenCV and ROS 1. (Dice number and color recognition, special symbology recollection, etc)}
            \resumeItem{Arranged environment running on specialized hardware to support the vehicle's perception system. (Nvidia Jetson TX2, Raspberry Pi)}
            \resumeItem{Worked on a platform to support the boat's shooting system using ROS2, modelling the system using Gazebo.}
        \resumeItemListEnd
    
  \resumeSubHeadingListEnd



%----------- WORK EXPERIENCE -----------

\section{Work Experience}
  \vspace{3pt}
  \resumeSubHeadingListStart
    
    \resumeSubheading
      {Freelance}{Remote}
      {General Purpose Programming \textbf{--} https://www.fiverr.com/eduarch42 }{Summer 2019, 2020, 2021, 2022}
        \resumeItemListStart
	    \resumeItem{Developed solutions in various fields, including Data Science, Backend Development, GUI Desktop Applications, with an overall 4.9/5 rating.}
            \resumeItem{Worked with clients from more than 10 countries around the world.}
        \resumeItemListEnd

    % \vspace{15pt}
    
  \resumeSubHeadingListEnd



%----------- AWARDS & ACHIEVEMENTS -----------

\section{Awards \& Achievements}
  \vspace{2pt}
  \resumeSubHeadingListStart
    \small{\item{
        \textbf{HackMTY 2022 Hackathon} \textbf{--} { $1^{st}$ Place in Chubb's sponsored challenge. } \\ \vspace{2pt}
        
	\textbf{Hackathon Banorte 2023} \textbf{--} { Finalist team. } \\ \vspace{2pt}
        
	\textbf{International Collegiate Programming Contest} \textbf{--} { Top 7 within ITESM MTY. } \\ \vspace{2pt}
        
	\textbf{Concurso Internacional de Ciencias} \textbf{--} { Finalist in Computer Science category.} \\ \vspace{2pt}
	
	\textbf{AWS Deep Racer} \textbf{--} { Participation in the AWS Deep Racer League and first generation at ITESM.}
    }}
  \resumeSubHeadingListEnd



%----------- PROJECTS -----------

% \section{Projects}
%     \vspace{3pt}
%     \resumeSubHeadingListStart
%       
%       \resumeProjectHeading
%         {\textbf{Filters and Fractals} $|$ \emph{\href{https://github.com/arasgungore/filters-and-fractals}{\color{blue}GitHub}}}{}
%           \resumeItemListStart
%             \resumeItem{A C project which implements a variety of image processing operations that manipulate the size, filter, brightness, contrast, saturation, and other properties of PPM images from scratch.}
%             \resumeItem{Added recursive fractal generation functions to model popular fractals including Mandelbrot set, Julia set, Koch curve, Barnsley fern, and Sierpinski triangle in PPM format.}
%           \resumeItemListEnd
%       
%     \resumeSubHeadingListEnd



%----------- SKILLS -----------

\section{Skills}
  \vspace{2pt}
  \resumeSubHeadingListStart
    \small{\item{
        \textbf{Programming:}{
	    \textit{Profiecient;} Python, Go.
	    \textit{Intermediate;} C, Rust, Javacript, Matlab.
	    \textit{Beginner;} C++} \\
	\vspace{3pt}
        
	\textbf{Technologies/Software:}{ Linux, Docker, GNU Debugger, (C)Make, Git, KiCAD, ROS[1,2], FreeCAD, OpenCV} \\ \vspace{3pt}
        
        \textbf{Languages:}{ Spanish (Native), English (B2 - PTEG), German (Elementary)}
        
        % \textbf{Frameworks}{: X, X, X} \\
        % \textbf{Developer Tools}{: X, X, X} \\
        % \textbf{Libraries}{: X, X, X} \\
        % \textbf{Applications}{: X, X, X}
    }}
  \resumeSubHeadingListEnd

% \vspace{2pt}

\setlength{\columnsep}{-10cm} % make columns more close to each other

\begin{center}
    \begin{multicols}{3}
	\qrcode[height=1.5cm]{https://www.linkedin.com/in/eduardohdz42} \vspace{1.2pt} \\ LinkedIn

	\qrcode[height=1.5cm]{https://bit.ly/46DgHb7} \vspace{1.2pt} \\ Github

	\qrcode[height=1.5cm]{https://bit.ly/43pkpT2} \vspace{1.2pt} \\ PDF
    \end{multicols}
\end{center}

\end{document}
